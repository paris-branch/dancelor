%% Counter of sets, such that they all have a unique id.
\newcounter{setcounter}

%% Define a file handle for an auxiliary file that will contain information
%% about the set pages in this run.
\newwrite\setpagesfile{}

%% At the beginning of the document, load \jobname.setpages's content for this document, and open it for writing. At the end, flush and close it.
\AtBeginDocument{%
  \IfFileExists{\jobname.setpages}{\input{\jobname.setpages}}{}%
  \immediate\openout\setpagesfile=\jobname.setpages%
}
\AtEndDocument{%
  \clearpage%
  \immediate\closeout\setpagesfile%
}

%% The \setpagesfile contains lines of the form `\setpageinfo{start}{2}{7}` or
%% `\setpageinfo{end}{4}{18}` expressing that set 2 starts at page 7 and set 4
%% ends at page 18. We define the function that writes those lines for the
%% current set, and we define the semantics of `\setpageinfo`: registering a
%% macro eg. `\def\setstart<n>{<page>}`.
\def\writethesetpageinfo#1{
  \write\setpagesfile{\string\setpageinfo{#1}{\thesetcounter}{\thepage}}%
}
\def\setpageinfo#1#2#3{%
  \expandafter\gdef\csname set#1#2\endcsname{#3}%
}

%% Blank pages will get a message to explain why they are blank. This macro does
%% nothing when the book is simple, and creates the “blank” page in question otherwise.
\def\blankpage{%
  \ifsimple%
  \else%
    \clearpage%
    \thispagestyle{norightmarknospecificity}%
    \null\vfill%
    \centering%
    \fbox{
      \begin{tabular}{@{}c@{}}
        This page is left blank\\
        for better alignment\\
        of the next multi-pages set.
      \end{tabular}%
    }%
    \vfill\vfill\null%
    \clearpage%
  \fi%
}

%% A macro returning the number of pages that the set will take. This is based
%% on the pages recorded in the previous run, as this information is not
%% available until after we have done the rendering.
\def\addblankpageifneeded{
  \ifcsname setstart\thesetcounter\endcsname
    \edef\thesetstart{\csname setstart\thesetcounter\endcsname}
    \edef\thesetend{\csname setend\thesetcounter\endcsname}
    \edef\thesetpages{\the\numexpr\thesetstart-\thesetend+1\relax}
    \ifnum\thesetpages=1
      %% One page works everywhere, so we do nothing.
    \else
      %% Two pages or more need to start on the left page.
      \ifodd\thepage
        \blankpage{}
      \else
        %% Things already start on the left page: we have nothing to do.
      \fi
    \fi
  \else
    %% We did not record page numbers: we have nothing to do in this run.
  \fi
}

%% ================================ [ API ] ================================= %%

%% To be called at the beginning of a set. Increases the set counter and records
%% the start page of the set both in `\thesetstartpage` and in the .setpages file.
\def\recordsetstartpage{
  \stepcounter{setcounter}%
  \writethesetpageinfo{start}%
  \addblankpageifneeded{}%
}

%% To be called at the end of a set. Adds a blank page if the previous run has
%% established that it is necessary and records the end page of the set.
\def\recordsetendpage{
  \writethesetpageinfo{end}%
}
